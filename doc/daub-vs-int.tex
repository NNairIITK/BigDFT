\documentclass[12pt]{article}

%\usepackage{pslatex}
%------------------------------------------------------------------------
\usepackage{amsmath,amsfonts}

%------------------------------------------------------------------------
%We start the document
\begin{document}

\section{Relation between Daubechies and Interpolating scaling functions in BigDFT code}
We know that the definition of an interpolating scaling function (ISF) of order $2m-1$ (with $2m-1$ vanishing moments\footnote{Note that the definition is in this case shifted by one wrt the definition we use in {\tt BigDFT} since for the ISF of the Poisson Solver the moments $M_\ell=\delta_{l,m}$, $\ell=0,\cdots m$.}) can be obtained from the autocorrelation of the Daubechies scaling function (DSF) of order $m$:
\begin{equation}
 \varphi(x) =\int \phi(t) \phi(t-x) {\rm d}t\;.
\end{equation}
In the {\tt BigDFT} code we use a formally different relation between the ISF and the DSF.
Indeed, a given function $f(x)$ can be expressend both in ISF or in DSF:
\begin{align}
 f(x)&= \sum_i f^D_i \phi_i(x) \notag \\
 &= \sum_i f^I_i \varphi_i(x)\;,
\end{align}
with the magic filter relation $f^D_i=\sum_j \omega_{i,j} f^I_j$. Hence
\begin{equation}
 f(x)=\sum_{i,j} \omega_{i,j} f^I_j \phi_i(x)\;.
\end{equation}
For the two relations to be compatible, one must have the filters $\omega$ to be optimal quadrature coefficients for the autocorrelation integral:
\begin{equation}
\sum_{i} \omega_{i,j} \phi_i(x) \simeq \int \phi(t) \phi(t-x) {\rm d}t\;.
\end{equation}
If this is true, we can have an exact relationship between DSF of order 8 and ISF of order 16 (in {\tt BigDFT} notation).
\end{document}

























